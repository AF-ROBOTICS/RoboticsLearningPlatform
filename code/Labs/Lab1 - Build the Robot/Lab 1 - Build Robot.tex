\documentclass{handout}

%\SetInstructor{Capt Steven Beyer}
\SetCourseTitle{ECE210: Introduction to Electrical and Computer Engineering}
\SetHandoutTitle{Lab 1 - Build DFECBot}

%\ShowAllBlanks

\usepackage[obeyspaces]{url}
\input{../../arduinoLanguage.tex}

%\showsoln \setsolncolor{red}

\lstset { %
	language=Arduino,
	frame=none,
	basicstyle=\footnotesize,
}

\begin{document}
	%	\footnotetext{Examples are abstracted from Tutorials Point, "Arduino", 2019, accessed Feb 18, 2019. [Online]. Available: https://www.tutorialspoint.com/arduino/index.htm}
	\maketitle
	
	\section{Collaboration Policy.}
	Collaboration is authorized.
	
	\textbf{Authorized Resources:} You may use any electronic or hard copy resource at your disposal as long as 1) you cite your sources and 2) your use of the materials does not go against the intent of the assignment. For example, you can use a software library that you find online to help develop a project if you cite where you found it. However, you cannot complete your project by copying all of the source code, schematics, etc and simply running it on the required hardware.
	
	\section{Objectives.} 
	\begin{enumerate}
		\item Build and test your DFECBot.
	\end{enumerate}
	
	\section{Materials.}
	\begin{enumerate}
		\item Soldered Printed Circuit Board (PCB)
		\item Arduino Uno
		\item 3D-printed structural pieces (A-D from left to right in the image below, plus the line sensor bar); The arrow in the image below shows the forward orientation of the robot as referred to in the rest of the lab
		\item 2 DC motors
		\item 2 wheels
		\item 2 tires (note left and right are different)
		\item Battery pack
		\item Jumper wires
		\item Switch
		\item 3 IR sensors
		\item 1 QTR-8RC Reflectance Sensor Array
		\item 1 1 x 11 90$^O$ header
		\item 8 hex standoffs (4 short and 4 long)
		\item Bolts
		\begin{enumerate}
			\item 6 - \# 4-40 x 1"
			\begin{enumerate}
				\item 4 for motors
				\item 2 for connecting A and B
			\end{enumerate}
			\item 8 - \# 4-40 x 1/4"
			\begin{enumerate}
				\item 8 for connecting standoffs to layers
			\end{enumerate}
			\item 10 - \# 2-56 x 1/4"
			\begin{enumerate}
				\item 2 for line sensors
				\item 6 for IR sensors
				\item 2 for wheels
			\end{enumerate}
		\end{enumerate}
		\item Nuts
		\begin{enumerate}
			\item 6 - \# 4-40
			\item 8 - \# 2-56
		\end{enumerate}
	\end{enumerate}
	
	\section{Layer A - Base Layer.}
	\begin{enumerate}
		\item Place the PCB on the printed spacers with the "Beyer" printed text towards the rounded edge.
		\item Solder the 1 x 11 90$^o$ header to the reflectance sensor array with the bent header opposite of the line sensors
		\item Secure the reflectance sensor array to the line sensor mount using 2 - \#2-56 x 1/4" bolts and nuts.
	\end{enumerate}
	
	\section{Layer B - Motor Layer.}
	\begin{enumerate}
		\item Use 4 - \# 4-40 x 1/4" bolts and attach the 4 short hex standoffs to Layer B (outermost holes and opposite of the printed B).
		\item Install the 2 DC motors to Layer B with wires facing inwards. The motors have yellow plastic cases that should be oriented towards the rear of Layer B. The white plastic drive shafts should be protruding through the side holes of Layer B. Use 4 - \# 4-40 x 1" bolts and nuts to secure each motor (Ensure the screw heads are to the outside of Layer B).
		\item \textbf{IR Sensor:} Connect three IR sensors to the sensor mounts on the front of the bot using 6 - \#2-56 x 1/4" bolts and nuts.
		\item Solder the male ends of two male-to-female jumper wires to the connectors on the switch.
		\item Use the nut and lock washer that came attached to the switch to attach it to Layer B in the same direction as the standoffs.
		\item Attach the wheels by sliding them onto the axles. Use 2 - \# 2-56 x 1/4" bolts to secure the wheels onto the axles.
		\item \textbf{Testing PCB:} Wire the motors, Arduino, battery, and switch to the PCB as shown in Figure. Download the ``PCB\_test" folder from , and open \textit{PCB\_test.ino} file. Compile and upload to the Arduino. If working correctly, the tires should go forward for 1 second, then left for 1 second, then right for 1 second, then backwards for 1 second. Disconnect the PCB from the Arduino and battery.
		\item Connect a male to female jumper to the PCB and feed through a slot in Layer B. These wires will connect to the battery and enable the removal of the battery without taking apart the bot.
		\item Secure the battery to Layer B using velcro.
	\end{enumerate}

	\section{Layer C - Arduino Layer.}
	\begin{enumerate}
		\item Connect Layer C to Layer B using the larger standoffs.
		\item Place the Arduino on the printed spacers (will only go one direction).
		\item Wire the Arduino to the PCB going through the slots in Layer C and B.
	\end{enumerate}

	\section{Connect Layer A to Layer B/C.}
	\begin{enumerate}
		\item Connect the three layers using the line sensor mount and 2 - \# 4-40 x 1" bolts and nuts.
	\end{enumerate}

	\section{Testing.}
	\begin{enumerate}
		\item Connect the battery to the jumper and turn on the PCB. The robot should drive in the same pattern as before.
	\end{enumerate}
		
		
	
	\newpage
	\clearpage
	\pagebreak
	
\end{document}
