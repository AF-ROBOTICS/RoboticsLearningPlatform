\documentclass{handout}

%\SetInstructor{Capt Steven Beyer}
\SetCourseTitle{ECE210: Introduction to Electrical and Computer Engineering}
\SetSemester{Spring 2019}
\SetHandoutTitle{Robot Line Following}

%\ShowAllBlanks

\usepackage[obeyspaces]{url}
\input{../arduinoLanguage.tex}

\graphicspath{{./figs/}}

%\showsoln \setsolncolor{red}

\lstset { %
	language=Arduino,
	frame=none,
	basicstyle=\footnotesize,
}

\begin{document}
	\maketitle
		\begin{figure}[H]
		\centering
		\includegraphics[width=.75\textwidth]{Cover.PNG}
	\end{figure}
	
	\section{Objectives.} 
	\begin{enumerate}
		\item Become familiar with the QTR-8RC Line Sensor Array.
		\item Utilize the Arduino to detect lines.
		\item Integrate the Line Sensors with the DFECBot and program the DFECBot to follow a line.
	\end{enumerate}
	
	\section{Materials.}
	\begin{enumerate}
		\item 1x QTR-8RC Reflectance Sensor Array
		\item DFECBot
		\item USB Programming Cable
		\item Black Tape
	\end{enumerate}
	
	\section{Introduction.}
	
	\subsection{QTR-8RC Reflectance Sensor Array}
	The QTR-8RC Line Sensor Array uses 8 infrared radiation (IR) reflectance sensor with an IR light-emitting diode (LED) and an IR sensitive phototransistor.\footnote{Information found at Pololu, \url{https://www.pololu.com/product/961}} The sensors are powered using the $3.3\ V$ and ground pins on the Arduino. Powering the sensors illuminates the IR LED. A $100\ \Omega$ resistor is on-board and placed in series with the LED to limit current. The output of the phototransistor is tied to a $10\ nF$ capacitor. The faster the capacitor discharges, the more reflective the surface is.\footnotemark[1] The QTR sensors have an optimal sensing distance of $3\ mm$.
	
	Wire your QTR-8RC Reflectance Sensor Array to the Arduino according to the schematic shown in Figure~\ref{Fig Line}.
	
	\begin{figure} [H]
		\centering
		\includegraphics[width=.5\textwidth]{line.jpg}
		\caption{QTR-8RC Reflectance Sensor Array Wiring Schematic}
		\label{Fig Line}
	\end{figure}
	
	\subsection{Example Code}
	Copy the Arduino sketch folder \path{robot_linefollowing}  from \textbf{Teams} (\path{Labs/robot_linefollowing}). Copy the \path{QTR-8RC} library from \textbf{Teams} (\path{Labs/Custom Libraries/QTR-8RC}) to your documents folder: \path{Documents/Arduino/libraries/}. Open the \path{robot_wallfollowing.ino} sketch.
	
	\subsection{QTR-8RC.h}
	The Arduino Sketch utilizes a library called QTR-8RC. This is a custom library adapted from the QTRSensors library to utilize the QTR-8RC Reflectance Sensor Array. This library provides functions to read values from the sensor array and return the distance off the line.
	
	\subsubsection{robot\_linefollowing.ino}
	This example Arduino Sketch provides code to calibrate the DFECBot's QTR-8RC Reflectance Sensor Array and determine the distance off a black line.
	
	\begin{lstlisting}
  // read calibrated sensor values and obtain a measure of the line position
  // with a position of 0.1 mm relative to the center of the line 
  // -33.2 mm to 33.2 mm
  int16_t position = qtr.readLineBlack(sensorValues);
	\end{lstlisting}
	
	\section{Procedure}
	\textbf{Use the example code provided, \path{robot_linefollowing}, \path{Motor.h}, and \path{QTR-8RC.h} to code the DFECBot to do the following:}
		
	\begin{enumerate}
		\item Use a ruler to confirm the accuracy of the line sensor read function
		\item Program the DFECBot to follow a line
	\end{enumerate}
	{\large \textbf{\underline{HINT:}}} You should remove all print statements and delays when testing your line following.
	\section{Notes on Proportional Controller:}
\end{document}
