\documentclass{handout}

%\SetInstructor{Capt Steven Beyer}
\SetCourseTitle{ECE210: Introduction to Electrical and Computer Engineering}
\SetHandoutTitle{Robot Motors}

%\ShowAllBlanks

\usepackage[obeyspaces]{url}
\input{../arduinoLanguage.tex}

\graphicspath{{./figs/}}

%\showsoln \setsolncolor{red}

\lstset { %
	language=Arduino,
	frame=none,
	basicstyle=\footnotesize,
}

\begin{document}
	\maketitle
	
	\section{Objectives.} 
	\begin{enumerate}
		\item Understand the fundamentals of Pulse Width Modulation (PWM) including period, frequency, duty cycle, and pulse width.
		\item Understand how PWM can be used to control the speed of motors.
		\item Demonstrate understanding by manipulating the motors at different speeds.
	\end{enumerate}
	
	\section{Pre-lab - To be done prior of class.}
	\subsection{Reading:}
	Adapted from \underline{Embedded Systems: Real-Time Interfacing to the MSP432 Microcontroller}, Jonathan W. Valvano, Third Edition, 2019.
		
	One effective way to deliver power to a DC motor in a variable manner is to use pulse width modulation (PWM). The basic idea of PWM is to create a \textbf{digital} output wave of fixed frequency, but allow the microctonroller to vary its duty cycle. The system is designed in such a way that \textbf{High+Low} is constant (meaning frequency/period is fixed). The \textbf{duty cycle} is defined as the fraction of time the signal is high:
	
	\[
		duty\ cycle = \frac{High}{High + Low} = \frac{High}{Period}
	\]
	
	Hence, duty cycle varies from 0 to 1 and will often be referred to as a percent. We interface this digital output wave to an external actuator (like a DC motor), such that power is applied to the motor when the signal is high, and no power is applied when the signal is low. We purposely select a frequency high enough so the DC motor does not start/stop with each individual pulse, but rather responds to the overall average value of the wave.
	
	\subsection{Key Terms:}
	\begin{itemize}
		\item \textbf{Pulse Width Modulation (PWM)}: A technique to deliver a variable signal (power) using an on/off signal with a variable percentage of time the signal is on (duty cycle).
		\item \textbf{Period/Frequency}: Time required to complete one full cycle (on/off). Frequency is the inverse of period and is constant. The Arduino pins used by the robot operate at 980 Hz.
		\item \textbf{Duty Cycle}: Percentage of time a digital signal is \textbf{on} over a period of time
	\end{itemize}


	
	\begin{figure} [H]
	\centering
	\includegraphics[width=.5\textwidth]{Figure1.PNG}
\end{figure}	
	
	\subsection{Examples:}
	\begin{enumerate}
		\item Pulsing an LED on and off.

	\begin{minipage}{.4\textwidth}
	You have already applied PWM to pulse the red LED using the Arduino. You varied the duty cycle through the use of the \mbox{\textit{analogWrite()}} function within the \textit{for-loop} which varied the power applied to the LED and consequently, the brightness. The code you used is to the right.
	\end{minipage}\hfill
	\begin{minipage}{.5\textwidth}
	\begin{figure} [H]
		\centering
		\includegraphics[width=\textwidth]{Figure2.PNG}
	\end{figure}
\end{minipage}

	\item Answer the following given the example shown below in Figure 1.
\begin{enumerate}
	\item What is the period of the signal?\soln{.2in}{100 ns}
	\item What is the pulse width of the signal (how long is the signal \textbf{HIGH})?\soln{.2in}{25 ns}
	\item What is the duty cycle?\soln{0in}{25\%}
\end{enumerate}

	\begin{figure} [H]
	\centering
	\includegraphics[width=.75\textwidth]{Figure3.PNG}
	\caption{}
\end{figure}

	\item Given that the \textit{analogWrite(pin, value)} function's \textit{value} parameter is an integer that represents the duty cycle between 0 (always off) and 255 (always on), what would be the \textit{value} passed to provide a 75\% duty cycle to the motors?
	
	\soln{.25in}{$75\% of 255 = 191.25 = 191$}
	
		\end{enumerate}
	\newpage
	\clearpage
	\pagebreak
	
	\subsection{Code:}
	\begin{enumerate}
		\item Copy the Arduino sketch fold \path{robot_motors} from \textbf{Teams} (\path{Labs/robot_motor})
		\item Open the \path{robot_motors} sketch.
		\item In the \textit{loop()} function, update the five calls to \textit{Motor\_Forward()} to drive the motor at the designated duty cycles. 
		\begin{itemize}
			\item The example provided, \textit{Motor\_Forward(50, 50);}, will drive both the left and right motors at (50/255)\% of the duty cycle. This is equivalent to a duty cycle of about 20\%.
		\end{itemize}
		\item Upload the sketch to the robot. Both motors of the robot should now operate at 6 different speeds (from 20\% duty cycle up to 100\% and back down to 15\%).
	\end{enumerate}
	
	\section{Lab.}
	This lab will demonstrate how varying the PWM signal changes the average power delivered to the DC motors and, effectively, the speed of the motors.
	
	\subsection{Materials:}
	
	\begin{enumerate}
		\item Assembled BeyerBot
		\item Arduino programming cable
		\item Laptop
		\item Male-to-female/male Y-connector and male-to-male wire (provided for you in the Moku:Lab)
	\end{enumerate}

	\subsection{Connect the Moku:Lab probes:}
	\begin{enumerate}
		\item Disconnect \textbf{ONLY} the wire from \textit{Pin-5} on the Arduino.
		\item Insert male end of the Y-connector into \textit{Pin-5}.
		\item Connect the male end of the wire coming from your Printed Circuit Board (the one disconnected from \textit{Pin-5}) into the female end of the Y-connector.
		\item Connect the positive (red) end of the Moku:Lab probe to \textit{Pin-5} on the Arduino using the Y-connector.
		\item Insert a male-to-male wire into the \textit{GND-Pin} on the Arduino and connect the ground (black) end of the Moku:Lab probe to the other end of the wire.
		\item Your setup should now look like Figure 2. All other wires should still be connected as they were previously.
		\begin{figure} [H]
			\centering
			\includegraphics[width=.75\textwidth]{Figure4.PNG}
			\caption{}
		\end{figure}
	\end{enumerate}
	
	\subsection{Measuring Pulse Width Modulation Signals:}
	\begin{enumerate}
		\item Place your robot on the Digital Multi Meter (DMM) (to allow the program to run without your robot driving away)
		\item Your Moku:Lab display should look like Figure 3. If it does not, raise your hand.
		\begin{figure} [H]
			\centering
			\includegraphics[width=.75\textwidth]{Figure5.PNG}
			\caption{}
		\end{figure}
		\newpage
	\clearpage
	\pagebreak
	
		\item Open the measurements bar by pressing the ruler in the bottom right of your screen.
		\begin{figure} [H]
			\centering
			\includegraphics[width=.75\textwidth]{Figure6.PNG}
		\end{figure}

		\item Add three measurement tools: Frequency, Period, and Pulse width. Press the ``plus" sign within the measurements toolbox to add a new tool. Select the tool to change it. Repeat this for each of the four tools.
		\begin{figure} [H]
			\centering
			\includegraphics[width=.75\textwidth]{Figure7.PNG}
		\end{figure}
	\begin{figure} [H]
		\centering
		\includegraphics[width=.75\textwidth]{Figure8.PNG}
	\end{figure}
	\begin{figure} [H]
	\centering
	\includegraphics[width=.75\textwidth]{Figure9.PNG}
\end{figure}
\item Turn on your robot and run the previously uploaded code.
\item What do you notice?

	\newpage
\clearpage
\pagebreak


\item Comment out all of the code within \textbf{\textit{loop()}} except for the portion that runs the motors at 75\% duty cycle (\textit{Motor\_Forward(191,191)}; \textit{delay(2000);}) and answer the following questions:
\begin{enumerate}
	\item What is the frequency of the signal? \soln{.3in}{980 Hz}
	\item What is the period of the signal? \soln{.3in}{}
	\item What is the pulse width of the signal (how long is the signal \textbf{HIGH})? \soln{.3in}{}
	\item Using these values, calculate the duty cycle. \soln{.3in}{}
	\item Is this duty cycle what you expected? \soln{.3in}{}
	\item Add a Duty cycle measurement tool on the display (reference step 4.3-4). Does the value match what you calculated?\soln{.3in}{}
	\item How does the speed of the motors correlate to the duty cycle? \soln{.3in}{}
	
\end{enumerate}
	\end{enumerate}
\section{Feedback.}
\begin{enumerate}
	\item Was there any part of the lab that was confusing or unclear? \soln{.3in}{}
	\item Is there any aspect of PWM that you still find confusing or would like more clarification on? \soln{.3in}{}
	\item What would you change from this lab for next semester? \soln{.3in}{}
\end{enumerate}
	
\end{document}
