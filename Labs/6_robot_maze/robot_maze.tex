\documentclass{handout}

%\SetInstructor{Capt Steven Beyer}
\SetCourseTitle{ECE210: Introduction to Electrical and Computer Engineering}
\SetHandoutTitle{Robot Maze}

%\ShowAllBlanks

\usepackage[obeyspaces]{url}

\graphicspath{{./figs/}}

%\showsoln \setsolncolor{red}

\begin{document}
	%	\footnotetext{Examples are abstracted from Tutorials Point, "Arduino", 2019, accessed Feb 18, 2019. [Online]. Available: https://www.tutorialspoint.com/arduino/index.htm}
	\maketitle
	\begin{figure}[H]
		\centering
		\includegraphics[width=.75\textwidth]{Cover.PNG}
	\end{figure}
	
	\section{Objectives.} 
	\begin{enumerate}
		\item Combine previous labs into a system that solves a complex task. 
		\item Troubleshoot ECE applications.
		\item Combine multiple systems (e.g., line following and wall detection) to build a more complex system.
	\end{enumerate}
	
	\section{Materials.}
	\begin{enumerate}
		\item DFEC Robot
		\item Maze
	\end{enumerate}
	
	\section{Robot Maze - Level 1: Line Following}
	The first level of the maze requires the robot to follow a white line to the end of the maze.
	
	\section{Robot Maze - Level 2: Advanced Line Following}
	The second level of the maze requires the robot to follow a white line to the end of the maze, detect the wall, turn around, and follow the line to the beginning of the maze, detect the wall, turn around, and stop.
	
	\section{Robot Maze - Level 3: Navigate Maze}
	The third level of the maze requires the robot to navigate the maze until it reaches the goal in the middle of the maze. It does not need to do anything special once it reaches the goal. Reaching the goal is enough.
	
	\section{Competition}
	The best times for each level will be recorded and displayed on a plaque in the ECE210 room.
\end{document}
