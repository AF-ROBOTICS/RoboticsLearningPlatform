\documentclass{handout}

%\SetInstructor{Capt Steven Beyer}
\SetCourseTitle{ECE210: Introduction to Electrical and Computer Engineering}
\SetSemester{Spring 2019}
\SetHandoutTitle{Lab 10 - Line Following}

%\ShowAllBlanks

\usepackage[obeyspaces]{url}
\input{../arduinoLanguage.tex}

%\showsoln \setsolncolor{red}

\lstset { %
	language=Arduino,
	frame=none,
	basicstyle=\footnotesize,
}

\begin{document}
	\maketitle
	
	\section{Collaboration Policy.}
	This is a team-of-two or individual laboratory. You may use any of the authorized resources listed below. DO NOT copy anyone else’s work.
	
	\textbf{Authorized Resources:} You may use any electronic or hard copy resource at your disposal as long as 1) you cite your sources and 2) your use of the materials does not go against the intent of the assignment. For example, you can use a software library that you find online to help develop a project if you cite where you found it. However, you cannot complete your project by copying all of the source code, schematics, etc and simply running it on the required hardware.
	
	\section{Documentation.}
	\textit{Date:}
	
	\textit{Instructor who helped me:}
	
	\textit{Help received:}
	
	\section{Objectives.} 
	\begin{enumerate}
		\item Become familiar with the QRE1113 Line Sensor.
		\item Utilize the Arduino to detect lines.
		\item Integrate the Line Sensors with the DFECBot and program the DFECBot to follow a line.
	\end{enumerate}
	
	\section{Materials.}
	\begin{enumerate}
		\item 3x QRE1113 Line Sensors
		\item 9x Female-to-Male Jumpers
		\item DFECBot
		\item USB Programming Cable
		\item Black Tape or a Black Marker an Paper
	\end{enumerate}
	
	\newpage
	\clearpage
	\pagebreak
	
	\section{Introduction.}
	
	\subsection{QRE1113 Line Sensor}
	The QRE1113 Line Sensor uses an infrared radiation (IR) reflectance sensor with an IR light-emitting diode (LED) and an IR sensitive phototransistor.\footnote{Information found at Sparkfun, \url{https://www.sparkfun.com/products/9454}} The sensors will be powered using the $5\ V$ and ground pins on the Arduino. Powering the sensors illuminates the IR LED. A $100\ \Omega$ resistor is on-board and placed in series with the LED to limit current. The output of the phototransistor is tied to a $10\ nF$ capacitor. The faster the capacitor discharges, the more reflective the surface is.\footnotemark[1] The QRE1113 has an optimal sensing distance between $.25\ mm$ and $5\ mm$.\footnote{Fairchild Semiconductor, QRE1113 Datasheet, \url{https://www.robotshop.com/media/files/zip/documentation-rob-09453.zip}}
	
	Connect your three QRE1113 Line Sensors to the spacer underneath the base layer. Space them evenly and secure them using three $\#2-56\ x\ 1/4”$ bolts and nuts. Wire the sensors according to the schematic shown in Figure~\ref{Fig Line}.
	
	\begin{figure} [H]
		\centering
		\includegraphics[width=.75\textwidth]{linefollowing.eps}
		\caption{QRE1113 Line Sensor Wiring Schematic}
		\label{Fig Line}
	\end{figure}
	
	\subsection{Example Code}
	Browse to \path{K:\DF\DFEC\ECE210\Labs\Lab 10 - Line Following} and copy the \path{robot_linefollowing} folder to your computer. Open \path{robot_linefollowing.ino}. Opening the \textit{.ino} file should open 3 files in your Arduino IDE: \path{robot_linefollowing.ino}, \path{drive.h}, and \path{TB6612FNG.h}. The two \textit{.h} files are the same header files used during Lab 9 and the \textit{.ino} file is your Arduino sketch.
	
	\subsubsection{robot\_linefollowing.ino}
	This example Arduino Sketch provides code to read the values from the DFECBot's left QRE1113 Line Sensor (see below example).
	
	\begin{lstlisting}
		// read and print value from the DFECBot's left line sensor
		float line_L = analogRead(lineL);
		Serial.print("Left: "); Serial.println(line_L);
	\end{lstlisting}
	
	\newpage
	\clearpage
	\pagebreak
	
	\section{Procedure}
	\textbf{Use the example code provided, \path{TB6612FNG.H}, and \path{drive.h} to code the DFECBot to do the following:}
		
	\begin{enumerate}
		\item Print the values from the DFECBot's center and right QRE1113 Line Sensors to the serial monitor.
		\item Observe how these values change if the sensor is over a solid black line.
		\item Program the DFECBot to follow a line using the 3 QRE1113 Line Sensors.
			\begin{enumerate}
				\item Drive forward.
				\item If the right sensor detects the line, turn right slightly until the middle sensor detects the line.
				\item If the left sensor detects the line, turn left slightly until the middle sensor detects the line.
			\end{enumerate}
		
			\textbf{Hint:} You may need to code two functions in \path{drive.h} to have the DFECBot make a slight right or left correction.
	\end{enumerate} 	
\end{document}
